\documentclass[twocolumn, 10pt, conference]{IEEEtran}

% packages start
\usepackage{graphicx}
\usepackage[caption=false,font=footnotesize]{subfig}
\usepackage{algorithm}
\usepackage{algorithmic}
\usepackage{listings}
\lstset{language=Python}
\usepackage{amsmath}
\usepackage{fixltx2e} % ensures float ordering is preserved
\usepackage{url}
\usepackage{breqn}
\usepackage{cite} % orders multiple citations
\usepackage{booktabs}

\usepackage{balance}
\usepackage{lipsum}

%\usepackage{hyperref} % citations and references are hyperlinks

%\usepackage[table,xcdraw]{xcolor} % colors on tables

\usepackage{siunitx}

% the following lines cause
%\usepackage{enumitem}
%\setlist{noitemsep,topsep=2pt,parsep=2pt,partopsep=0pt}


\IEEEoverridecommandlockouts

\begin{document}

\hyphenation{VA-NET}
\hyphenation{VA-NETs}
\hyphenation{pa-th--fin-ding}
\hyphenation{heu-ris-tic--dri-ven}
\hyphenation{match-es}
\hyphenation{rang-es}
\hyphenation{mea-sure-ments}
\hyphenation{ge-o-graph-ic}
\hyphenation{to-pol-o-gy-based}



%
% --- Author Metadata here ---
%\conferenceinfo{SIGMETRICS2016}{2016 Juan-les-Pins, France}
%\CopyrightYear{2016} % Allows default copyright year (20XX) to be over-ridden - IF NEED BE.
%\crdata{0-12345-67-8/90/01}  % Allows default copyright data (0-89791-88-6/97/05) to be over-ridden - IF NEED BE.
% --- End of Author Metadata ---

\title{Experimental study on round trip times of web applications}

%
% You need the command \numberofauthors to handle the 'placement
% and alignment' of the authors beneath the title.
%
% For aesthetic reasons, we recommend 'three authors at a time'
% i.e. three 'name/affiliation blocks' be placed beneath the title.
%
% NOTE: You are NOT restricted in how many 'rows' of
% "name/affiliations" may appear. We just ask that you restrict
% the number of 'columns' to three.
%
% Because of the available 'opening page real-estate'
% we ask you to refrain from putting more than six authors
% (two rows with three columns) beneath the article title.
% More than six makes the first-page appear very cluttered indeed.
%
% Use the \alignauthor commands to handle the names
% and affiliations for an 'aesthetic maximum' of six authors.
% Add names, affiliations, addresses for
% the seventh etc. author(s) as the argument for the
% \additionalauthors command.
% These 'additional authors' will be output/set for you
% without further effort on your part as the last section in
% the body of your article BEFORE References or any Appendices.

% double-blind submission

\author{
Rafi Khaled, Rui Meireles
\\ 
\small{\texttt{\{rkhaled, rui.meireles\}@vassar.edu}}
\\ \\
\small \IEEEauthorblockA{Department of Computer Science, Vassar College, USA}
}

\maketitle
%\thispagestyle{plain} % enables page numbers
%\pagestyle{plain} % enables page numbers

%\category{C.2.2}{Computer-Communication Networks}{Network Protocols}[Routing protocols]
%\keywords{Vehicular networks, Multi-hop, Routing, Forwarding}

\setlength{\belowdisplayskip}{10pt} \setlength{\belowdisplayshortskip}{10pt}
\setlength{\abovedisplayskip}{10pt} \setlength{\abovedisplayshortskip}{10pt}
\setlength{\textfloatsep}{10pt plus 1.0pt minus 2.0pt}
\setlength{\floatsep}{10pt plus 1.0pt minus 2.0pt}
\setlength{\intextsep}{10pt plus 1.0pt minus 2.0pt}

\begin{abstract}
% motivation, problem statement, approach, results
Replace by concrete abstract.
\end{abstract}


\section{Introduction}
\label{sec:introduction}

% The introduction is very important. By the time the reviewer finishes the 
% introduction, he probably has his mind made up already: he'll read the rest 
% of the paper looking for evidence to support his decision.

% what is the problem and why is it importantly
% why is the problem hard?
% what is wrong with existing solutions?
% what do we propose to do and why is it better (i.e. what are our contributions)?

Introduction.

Figure~\ref{fig:example} is an example figure.

\begin{figure}[t]
\centering
    \includegraphics[width=0.41\textwidth]{fig/example-fig}
     \caption{Example figure}
      \label{fig:example}
\end{figure}

\section{Related work}
\label{sec:related-work}
% what other works are related to ours, and how.
Related work.

\section{Methodology}
\label{sec:methodology}

% explain what we've done in order to ensure reproducibility

Collection of data was conducted using two Python libraries: pyping and PycURL. PycURL is a Python interface to libcurl, which is a client-side URL transfer library. PycURL allows one to fetch various objects identified by a URL: in our case, objects corresponding to the URLs of various websites hosted on servers across the world. [1]

First, we wanted to calculate the RTT as measured by a calculation involving, at the highest level, a website’s pretransfer time and time to first byte: the RTT was calculated as the pretransfer time subtracted from the time to first byte, both of which were obtained with PycURL. Knowing this, it is as simple as creating a new Curl Object from PycURL, setting the relevant options, i.e. the name of the website and the FOLLOWLOCATION to 1. Setting the FOLLOWLOCATION to 1 tells the library to follow any Location: header that the server sends as part of a HTTP header in a 3xx response. The Location: header can specify a relative or an absolute URL to follow. The library will issue another request for the new URL and follow new Location: headers all the way until no more such headers are returned. [2]


\section{Evaluation}
\label{sec:evaluation}

% present and analyze results.

Evaluation.

\section{Conclusions}
\label{sec:conclusions}

%\begin{itemize}
%\item Briefly restate the main points, results and contributions.
%\item Say what the lessons learned were
%\end{itemize}

% more synthesis than summary

Conclusions.



% The following two commands are all you need in the
% initial runs of your .tex file to
% produce the bibliography for the citations in your paper.
\bibliographystyle{IEEEtran}
%\small{
\bibliography{IEEEabrv,laspaper}
%\balancecolumns % must be placed in the laspaper.bbl right where you want the column break to occur
%}

%That's all folks!
\end{document}



